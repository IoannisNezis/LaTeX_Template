\documentclass[a4paper,dvipsnames]{scrartcl}
\usepackage{listings}
\lstset{language=Java}
\usepackage[utf8]{inputenc}
\usepackage[ngerman]{babel}
\usepackage{amsmath}
\usepackage{amssymb}
\usepackage{fancyhdr}
\usepackage{color}
\usepackage{graphicx}
\usepackage{lastpage}
\usepackage{listings}
\usepackage{tikz}
\usepackage{pdflscape}
\usepackage{subfigure}
\usepackage{float}
\usepackage{polynom}
\usepackage{hyperref}
\usepackage{tabularx}
\usepackage{forloop}
\usepackage{geometry}
\usepackage{listings}
\usepackage{fancybox}
\usepackage{tikz}
\usepackage{nicematrix}
\usepackage[T1]{fontenc}
\usepackage[utf8]{inputenc}
\usepackage{textcomp}
\usepackage{comment}
\usepackage{eurosym}
% \usepackage{ marvosym }
\usepackage{adjustbox}
\usepackage{listings}
\usepackage[noend]{algpseudocode}
\usepackage{stmaryrd}
\usepackage {dsfont}
\usepackage{tabstackengine} %polynomial division
\usepackage{forest} %trees
\usepackage{xcolor}
\usepackage{graphicx,amsmath,float,cancel,multirow,enumerate,ngerman}
\usepackage{tabstackengine} %polynomial division
\input kvmacros
\stackMath
\newsavebox\tempbox
\newlength\templen
\def\rl#1{%
  \sbox\tempbox{$#1$}%
  \setlength\templen{\wd\tempbox}%
  \llap{\rule{1.5pt}{.1ex}}\rule{\templen}{.1ex}\rlap{\rule{1.5pt}{.1ex}}}
\setstacktabulargap{0pt}
%Größe der Ränder setzen
\geometry{a4paper,left=3cm, right=3cm, top=3cm, bottom=3cm}

\fancypagestyle{firstpage}
{
%Kopf- und Fußzeile
  % 
  \fancyhead[L]{Tutor: \TUTOR}
  \fancyhead[C]{\COURSE}
  \fancyhead[R]{\today}
}
\pagestyle {fancy}
\fancyhead[L]{\COURSE}
\fancyhead[R]{Abgabe \NUMBER}
\fancyfoot[L]{}
\fancyfoot[C]{}
\fancyfoot[R]{Seite \thepage /\pageref*{LastPage}}
%Formatierung der Überschrift, hier nichts ändern
\def\header#1#2{
  \begin{center}
    {\Large Aufgabenblatt #1}\\
    {(Due #2)}
  \end{center}
}

%New colors defined below
\definecolor{codegreen}{rgb}{0,0.665,0}
\definecolor{codegray}{rgb}{0.5,0.5,0.5}
\definecolor{codepurple}{rgb}{0.58,0,0.82}
\definecolor{codeorange}{rgb}{1,0.5,0.25}
\definecolor{backcolourg}{rgb}{0.96,0.96,0.94}
\definecolor{backcolour}{rgb}{1,1,1}

%Code listing style named "mystyle"
\lstdefinestyle{mystyle}{
  backgroundcolor=\color{backcolour},   commentstyle=\color{codegreen},
  keywordstyle=\color{codeorange},
  numberstyle=\tiny\color{codegray},
  stringstyle=\color{codegreen},
  basicstyle=\ttfamily\footnotesize,
  breakatwhitespace=false,         
  breaklines=true,                 
  captionpos=b,                    
  keepspaces=true,                 
  numbers=left,                    
  numbersep=5pt,                  
  showspaces=false,                
  showstringspaces=false,
  showtabs=false,                  
  tabsize=2
}
\lstset{style=mystyle}

%Definition der Punktetabelle, hier nichts ändern
\newcounter{punktelistectr}
\newcounter{punkte}
\newcommand{\punkteliste}[2]{%
  \setcounter{punkte}{#2}%
  \addtocounter{punkte}{-#1}%
  \stepcounter{punkte}%<-- also punkte = m-n+1 = Anzahl Spalten[1]
  \begin{center}%
  \begin{tabularx}{\linewidth}[]{@{}*{\thepunkte}{>{\centering\arraybackslash} X|}@{}>{\centering\arraybackslash}X}
      \forloop{punktelistectr}{#1}{\value{punktelistectr} < #2} %
      {%
        \thepunktelistectr &
      }
      #2 &  $\Sigma$ \\
      \hline
      \forloop{punktelistectr}{#1}{\value{punktelistectr} < #2} %
      {%
        &
      } &\\
      \forloop{punktelistectr}{#1}{\value{punktelistectr} < #2 } %
      {
        &
      } &\\
      \end{tabularx}
  \end{center}
}
\newcommand{\N}{\mathbb{N}}
\newcommand{\R}{\mathbb{R}}
\newcommand{\Z}{\mathbb{Z}}
\newcommand{\C}{\mathbb{C}}

\newcommand{\toppage}{
\thispagestyle{firstpage}
\begin{tabularx}{\linewidth}{m{0.4 \linewidth}X}
  \begin{minipage}{\linewidth}
    \STUDENTA\\
    \STUDENTB\\
  \end{minipage} & \begin{minipage}{\linewidth}
    \punkteliste{1}{\EXERCISES}
  \end{minipage}\\
\end{tabularx}

\header{Nr. \NUMBER}{\DEADLINE}
\setlength{\parindent}{0em} 
}